\documentclass[runningheads,a4paper]{llncs}
\usepackage[utf8]{inputenc}
\usepackage{amssymb}
\setcounter{tocdepth}{3}
\usepackage{graphicx}

\usepackage{url}
\urldef{\mailsa}\path|{jjmerelo, bortiz}@ugr.es|    
\newcommand{\keywords}[1]{\par\addvspace\baselineskip
\noindent\keywordname\enspace\ignorespaces#1}

\begin{document}

\mainmatter  % start of an individual contribution

% first the title is needed
\title{DeepGraph State of the art }

% a short form should be given in case it is too long for the running head
\titlerunning{DeepGraph State of the art}

% Author's name 
\author{Juan Julián Merelo Guervós\and Bartolomé Ortiz Viso}


% the affiliations are given next; don't give your e-mail address
% unless you accept that it will be published
\institute{Higher Technical School of Information Technology and Telecommunications Engineering\\Granada University\\
\mailsa\\
\url{https://www.geneura.wordpress.com}}

\maketitle


\begin{abstract}
Abstract place
\keywords{Complex systems}
\end{abstract}


\section{Introduction}

A network is usually represented through a graph, that is, a set of
nodes and edges that express some type of relationship between those
nodes. We talk about complex networks \cite{strogatz2001exploring} when
these graphs have a variety of characteristics which imply that the
system they represent is a complex system: mainly, the existence of
power laws \cite{newman2005power} and the *small world* feature
\cite{watts1998collective}. In particular, we refer to social networks
when nodes or edges express groups or social relations
\cite{mislove2007measurement}. In general, social networks are often
complex networks, at least when they reach a certain size; due to,
during the growth of a social network, a phase change occurs that
makes these complex network features emerge. In this project we
propose to study complex networks with an unified approach and common
tools, so we will try to define exactly what is our main interest.

The fact that they behave as complex systems entails a series of
consequences with respect to their resistance to attacks
\cite{albert2000error} and growth  \cite{newman2005power}, as well as
the creation of mesostructures and common  \cite{milo2002network}
and transitions of phase \cite{milo2002network} preceded by a reached
critical state, generally achieve by self-organization. This state of
self-organized criticality \cite{bak2013nature} is precisely in the
vicinity of the phase changes that interest us in this research
project. Self-organization explains that many natural systems are in
this critical state which causes the existence of power laws, but also
other phenomena such as the so-called * pink noise * and long-distance
correlations between different events. 

The study of complex networks and all its features underwent a great
expansion since the end of the 90s, at the same time as the rise of
the World Wide Web. This boom was accompanied by a large number of
tools to analyze, visualize and calculate in graphs. With the
existence of a multitude of tools, networks have been used as a model
for a great variety of complex systems: genomics \cite{park2015deep},
cancer diagnosis \cite{roullier2010graph}, transport network models
\cite{ma2015} and traffic regulation systems with traffic lights
\cite{goel2017} or the connectome or human brain connection 
\cite{sporns2005human}; critical status and self-organizing models have
been detected in a large number of systems, from dating networks
\cite{leydesdorff2018discontinuities,benjafield2017between}, to software
development teams \cite{gorshenev2004punctuated}, both of which are two
acknowledged creation's processes whose criticality has also been
examined by different authors \cite{tadic2017mechanisms}; these processes
have also been discover in air pollution \cite{shi2009self} and network
traffic \cite{nagel1996network}. However, in general, there are no tools
to model, analyze, visualize and process these systems in a unified
way, neither in the same terminology, nor in the way to find, based on
the evidence or indications of emerging behavior, the underlying model
that allows a better understanding of the system or transfer the
models from an environment such as social networks to different
environments such as air pollution models or genomics; in works like
\cite{gd2012} different types of biological networks are reconstructed
using a series of autocorrelation and topological indexes. 


% <-- This was "This project's applicants". It will probably have to be
% expanded with all kinds of applications -->

Network theory has been applied to multiple fields. The first use of
complex networks in a system's  analysis was done on the network of
coauthors in the evolutionary computing's field \cite{redes-arxiv,ec-network-2006,merelo2007bce}, the network of web pages of the
companies of the IBEX \cite{ibex2015}, doctoral theses in different areas,
such as bibliometrics or blogs in that same area
\cite{TORRESSALINAS2011168}; even the network of passes of the Spanish
soccer team \cite{futbol2005,futbol2011}. In these two works,
computational tools and similar analysis' measures were used, but
there were several challenges that needed an ad hoc solution: the
extraction of data and the assignment of properties and elements of
the original systems to a graph on which we could carry out the
analysis. Any system has multiple elements, and although in a network
of passes it may seem obvious that the players must be the nodes and
pass the edges, other combinations, such as areas of the field or the
type of player, are a priori possible. The collection of information
is totally ad hoc, and includes heuristics to recognize players or
*scraping* tools to extract information from web pages. Team members
also have experience in the area of metabolomics \cite{granados2017p1}. 

More recent studies done by our group reveal certain network effects
too \cite{wikipedia17}, but without the complex network from which they
emerge being evident or clear. For example, in studies of code
repositories \cite{soc17} there are effects of self-organized criticality
even if there are only one, or a small number of users. It is
necessary, therefore, to carry out some type of systematic study of
different descriptions related to the system to choose the network
model that best explains the effects that are observed. In the study
on different repositories \cite{repo17}, effects of this organized
criticality are found, but the network models start from different
assumptions; however, in studies such as Herráiz et al. \cite{Herraiz08},
long-distance correlations are not found and this can be due to both
the representation of the model and the time scale considered. 

% <!-- This is the crux to the project, but I don't know if it belongs -->
% <!-- in the paper --> 

In general, there is not a \textit{natural} way to represent the underlying
network in a phenomenon or a system, in the same way that there is no
\textit{natural} way to represent the data of a problem to apply recognition
or prediction systems. However, the approach called \textit{Deep Learning}
\cite{lecun2015deep} creates different layers of processing the initial
data that allow the creation of a representation for the intermediate
problem, easier to tackle but above all, to be carried out
automatically.

% <-- Probably not good for the paper 
This approach is what we intend to apply for this project, but working
on reticular information (web-like shape) instead of working on tabular
information (table shape) which is what the Deep Learning approach
carries out; this approach continues from the one used in EphemeCH
project \cite{ECTA2015cotta} and focuses on one aspect, the graphs, the
same topic that will be treated in the future project:DeepBio
(provisionally conceded by the Ministry). In any case, the
optimization approach has been applied by our group to the
optimization of neural networks, a type of graph that behaves as a
complex network in some cases, in a line of research that began in the
early 90s \cite{glvq95} and has continued to the present \cite{castilloCEC99,parras2016radial}. The creation of several \textit{layers} is a project for
processing and optimization which is widely proposed in the field of
Deep Learning in ourdays, as it is done in jobs such as \cite{zhu2012},
where they stack several data dimensions with the object to jointly
analyze the metabolomic and transcriptomic networks of cellular
regulation. 

In this project we would try to apply that to the case of graphs and
the modeling of systems that can be represented by graphs. Moreover,
by applying Deep Learning to environments with large amounts of data,
new paradigms have been found that have allowed for much more advanced
classification and learning tasks than the previous generation's
ones. Our Deep Graph's approach also aims to encourage this type of
development.

% Up to here... -->

It is remarkable that there ir not a \textit{natural} way to visualize these
information too. Apart from the existing tools in the area such as
Gephi or Paje, new visualization methods have been proposed using
methods such as biplot \cite{torres2013use} or neural networks
\cite{blogtalk2jj} and others that can even reflect the evolution of a
complex graph \cite{weblogs-JNCA}; there are companies such as Graphext
dedicated to the visualization in graph mode of different quantities,
but in most cases they are ad hoc efforts and whose interpretation in
some cases is completely separate from the underlying graph model and,
therefore, does not allow to really evaluate different models or
compare them. The visualization methods are also computationally
complex and, in any case, may require additional optimization
techniques to be executed efficiently. 

In order to develop this approach, it is proposed to investigate both
optimization techniques and Deep Learning techniques to change the
representation and a system's model using graphs. There have already
been several researchers who have applied Deep Learning to graphs, for
example to predict traffic jams \cite{ma2015} or for prediction in genomic
networks \cite{park2015deep}, but have focused more on obviating the
reticular model to work directly on tabular data. Having the
underlying graph will allow not only making predictions or
classifications, but also having a more effective model with high
availability of the system that can be used as an assitance in
decision making. 

\bibliographystyle{acm}
\bibliography{deepgraph_no_spanish.bib}


\section*{Possible appendix}


\end{document}
